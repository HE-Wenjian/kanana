%\subsection{Challenges of Patching}
{\bf Challenges of Patching.}
\label{ssec:challenges}

% ``application'' is overloaded at this point -MEL
Requiring and encouraging the adoption of the latest security patches
is a matter of common wisdom and prudent policy.  It appears, however,
that this wisdom is routinely ignored in practice.  This disconnect
suggests that we should look for the reasons underlying users'
hesitancy to apply patches, as these reasons might be due to
fundamental technical challenges that are not yet recognized as such.
We believe that the current mechanics of applying patches prove to be
just such a stumbling block, and we contend that the underlying
challenges need to and can be addressed in a fundamental manner {\em
  by extending the core elements of the ABI and the executable file
  format}.

%The need for such fundamental changes will become apparent as we
%point out the problems facing the owners of software targeted by a
%patch.

Mission-critical systems seem hardest to patch.  They can ill afford
downtime, and the owner may be reluctant to patch due to the real or
perceived risk of the patch breaking essential functionality.  For
example, patching a component of a distributed system might lead to a
loss or corruption of state for the entire system.  An administrator
might also suspect that the patch is incompatible with some legacy
parts of the system.  Even so, the patch may target a latent
vulnerability in a software feature that is not now in active use, but
also cannot be easily made unreachable via configuration or module
unloading.  The administrator is forced to accept a particularly
thorny choice: inaction holds as much risk as a proactive
``responsible'' approach.  Since the risks of patching must
be weighed against those of staying unpatched, we seek to
%In this paper, however, we will not analyze such
%risk trade-offs, but rather concentrate on how we can 
{\bf shift the balance of this decision toward hot patching by making
  it not only possible, but also less risky in a broad range of
  circumstances}. We contend that this can only be done through good
engineering and making patching a part of the standard toolchain.

%Consider the following circumstances:
%\begin{itemize}
%  \item The patchee is part of a distributed system and cannot afford
%    loss of state, even though it can afford some very brief periods
%    of response delay ({\it e.g.,} its loss of state would require the
%    production distributed system to be restarted as whole).
%  \item The administrator suspects that the patch might not be
%    compatible with other parts of the system, and so non-applicable
%    within his legacy environment.
%  \item The patch targets an exploitable vulnerability in an
%    application feature that is not in active use but cannot be easily
%    made unreachable via the application's own configuration or module
%    support.
%\end{itemize}

Our key observation is that current binary patches, whether ``hot'' or
static, are almost entirely opaque and do not support any form of
reasoning about the impact of the patch (short of reverse engineering
both the patch and the targeted binary).  In particular, it is hard
for the software owner to find out whether and how a patch would
affect any particular subsystem or compatibility of the target with
other software in any other way than applying the patch on the test
system and trying it out, somehow finding a way to faithfully
replicate the conditions of the production environment.

Given these circumstances, our tool Katana and our Patch Object format
not only seek to make possible the mechanics of hot patching, but also
enable administrators to reduce the risk of applying a particular fix
by providing them with enough information to support examination of
the patch structure, reasoning\footnote{By which we mean manual,
  human-level reasoning, although applying automated reasoning methods
  is an interesting (and open) avenue of research.} about its
interaction with the rest of the system, and an understanding of the
tradeoffs involved in applying it.

%Thus the case for our suggested architecture is not only ``hot
%patching'' of running processes but also reducing the real or
%perceived risk though the ability to examine the structure of the
%patch and reason about its interaction with the software.

