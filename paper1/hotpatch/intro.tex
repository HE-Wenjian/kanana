\section{Introduction}
\label{sec:intro}
It is somewhat ironic that users and organizatations hesistate to
apply patches whose stated purpose is to support availability or
reliability precisely because the process of doing so can lead to
downtime (both from the patching process itself as well as
unanticipated issues with the patch).  Periodic reboots in desktop
systems --- irrespective of the vendor --- are at best annoying.
Reboots in enterprise environments ({\it e.g.,} trading, e-commerce,
core network systems), even for a few minutes, imply large revenue
loss or an extensive backup and failover infrastructure with rolling
updates.  

{\footnotesize
\begin{quote}
``Some reports, such as the case of the Conficker outbreak within 
Sheffield Hospital's operating ward, suggest that even security-
conscious environments may elect to forgo automated software patching, 
choosing to trade off vulnerability exposure for some perceived notion 
of platform stability...'' -- \url{http://mtc.sri.com/Conficker/}
\end{quote}
}

We question whether this {\em de facto} acceptence of
significant downtime and redundant infrastructure should not be
abandoned in favor of a reliable hot patching process.

%
% SB: This intro would fit in a larger version, but perhaps
%       not in this strapped-for-space one?
%
%As computing penetrates deeper into our modern infrastructure, devices
%(mobile or otherwise) and applications running on them are expected to
%be safer and more reliable compared to the previous generation. 
%SB: proposed variant:
%Even the requirement of periodic desktop reboots is nowadays
%perceived as annoying, whereas for the enterprise environments  
%taking down trading, popular e-commerce,
%or core network systems even for a few minutes implies huge
%revenue losses.
%
% Server
% uptimes are becoming ever more demanding, especially for systems
% hosting e-commerce sites, email services or social networking
% domains. Even personal computers are not spared the anathema of
% "necessary" reboots whenever an application or operating system
% service needs to be patched. Such frequent reboots interrupt both
% network-facing services and interactive applications, thereby
% frustrating users at both ends. More seriously, in the case of network
% switches and routers, disconnecting them from the network for the
% purposes of patching might render the underlying network unreachable
% in the absence of redundancy. Trading systems need to be active all
% through trading hours, and popular e-commerce sites suffer huge
% revenue losses when their servers are down for even a few minutes
% during the frenetic period of holiday shopping.
%\paragraph{}

% Despite all the foresight that goes into building what are potentially
% thought to be safe, reliable and scalable applications, such
% guarantees of security and availability cannot be proven or even
% sometimes convincingly relayed to the skeptical customer. The
% customer's trust in a system is seeded in a long-term relationship
% with the vendor whose track-record determines the product's ultimate
% success. We believe that all software is flawed, as all programmers
% are flawed. It is the magnitude of the flaw that separates the expert
% programmer from a mediocre one. As such, fixes or enhancements to
% released software is a foregone conclusion even as the original
% software is being deployed.
%
% SB: rephase and cut:

Software, the product of an inherently human process, remains a flawed
and incomplete artifact.  This reality leads to the uncomfortable
inevitability of future fixes, upgrades, and enhancements.  Given the
way such fixes are currently applied ({\it i.e.,} patch and reboot),
downtime is a foregone conclusion even as the software is released.

%Yet, all software is flawed, as all programmers are flawed. Thus
%the inevitability of fixes or enhancements to
%released software is a foregone conclusion even as the original
%software is being deployed.

%SB: small cuts
%\paragraph{}
%Such fixes or enhancements form what is routinely called a software
%patch or an upgrade. 

%arcane wasn't quite the right word, i think -MEL
%barbaric, perhaps? Immature?
While patches themselves are a necessity, we believe that the process
of {\it applying} them remains rather crude.  First, the target
process is terminated, the new binary and corresponding libraries (if
any) are then written over the older versions, the system is restarted
if necessary, and finally the upgraded application begins execution.
Besides the appreciable loss in uptime, all context held by the
application is also lost, unless the application had saved its
state to persistent storage~\cite{crashonly,brown02rewind} and
later restored it (which
%seldom happens). 
is expensive to design for, implement, and execute).  In the case of
mission-critical services, even after a major flaw is unveiled and a
patch subsequently created, administrators likely wish to apply the
patch and upgrade the process without actually restarting the program
and losing state and time.  This requirement serves as our motivation
for {\it hot patching}. We focus on systems which are both
high-availability and store significant state (which would be lost on
a restart) such as those found in the power grid.

