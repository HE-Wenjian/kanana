%\subsection{Why Not Just Employ Redundancy?}
{\bf Why Not Just Employ Redundancy?}
\label{ssec:redundancy}

Redundant infrastructure, containing replicas of nodes and service
paths, often helps an organization bridge the service disruption
stemming from patches.  We believe, however, that redundancy isn't
always the best approach for ensuring availability during an upgrade
or security-critical patching process.  Rather than an established
best practice, we invite the reader to see redundancy as
an extreme measure that needlessly duplicates hardware, networking, and
software of the original system. We suggest that redundancy is: 

%Such a mirrored system ensures that service remains
%uninterrupted whenever a {\it failover} is initiated. A {\it failback}
%scenario occurs when the redundant machine hands over control back to
%the original. 

\renewcommand{\labelenumi}{\alph{enumi}.}

\begin{enumerate}
\item {\bf expensive -} especially in medium-sized enterprises where
  the cost of a single server, gateway, or switch is high enough to
  outweigh the benefits of redundancy.

\item {\bf wasteful -} Redundant systems are typically passive
  bystanders, lying in wait for an active machine to
  initiate a failover.

\item {{\bf requires complicated logic -}} Transferring application
  state (even across multiple homogenous systems) is non-trivial,
  especially when the state transfer occurs within hardware (such as
  for call trunks).

\item {{\bf specialized -}} The process of building system redundancy
  is not easily generalizable across heterogenous systems and requires
  full knowledge of the underlying protocol and application state in
  order to provide faithful failover and failback.
\end{enumerate}

It should be noted that the last two items apply even for virtualized
redundant systems, which often do not have the traditional overhead of
redundant hardware.  We do not claim that redundancy does not have its
place, but redundancy does not provide the easy, ubiquitous solution to
high-availability stateful applications which we hope to provide
through patching

% LocalWords:  stateful virtualized
