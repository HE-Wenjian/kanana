% Created 2010-05-24 Mon 14:34
\documentclass[11pt]{article}
\usepackage[utf8]{inputenc}
\usepackage[T1]{fontenc}
\usepackage{graphicx}
\usepackage{longtable}
\usepackage{float}
\usepackage{wrapfig}
\usepackage{soul}
\usepackage{amssymb}
\usepackage{hyperref}


\title{Katana: A Userland Toolchain-Oriented Hotpatching System}
\author{James Oakley}
\date{24 May 2010}

\begin{document}

\maketitle

\setcounter{tocdepth}{3}
\tableofcontents
\vspace*{1cm}

\section{Introduction}
\label{sec-1}

  
\section{What Katana Does}
\label{sec-2}

\section{What Katan Does Not Do (Yet)}
\label{sec-3}

\section{How to Use Katana}
\label{sec-4}

  Katana is intended to be used in two stages. The first stage
  generates a patch object from two different versions of an
  treee. By an object tree, we mean the set of object files (.o files)
  and the executable binary they comprise. Katana works completely at
  the object level, so the source code itself is not strictly
  required, although all objects must be compiled with debugging
  information. This step may be done by the software vendor. In the
  second stage, the patch is applied to a running process. The
  original source trees are not necessary during patch application, as
  the patch object contains all information necessary to patch the
  in-memory process at the object level. It is also possible to view
  the contents of a patch object in a human-readable way for the
  purposes of sanity-checking, determining what changes the patch
  makes, etc.
\subsection{Preparing a Package for Patching Support}
\label{sec-4.1}

   Katana aims to be much less invasive than other hot-patching system
   and require minimal work to be used with any project. It does,
   however, have some requirements.\\
   Required CFLAGS:
\begin{itemize}
\item -g
\end{itemize}
   Recommended CFLAGS:
\begin{itemize}
\item -ffunction-sections
\item -fdata-sections
\end{itemize}
   Recommended LDFLAGS:
\begin{itemize}
\item --emit-relocs
\end{itemize}
\subsection{To Generate a Patch}
\label{sec-4.2}

\subsection{To Apply a Patch}
\label{sec-4.3}

\subsection{To View a Patch}
\label{sec-4.4}

\subsection{See Also}
\label{sec-4.5}

   the katana manpage
\section{Patch Object Format}
\label{sec-5}

\section{Patch Generation Process}
\label{sec-6}

\section{Patch Application Process}
\label{sec-7}

\section{Roadmap}
\label{sec-8}


\end{document}